% LTeX: language=de-DE
\pdfpcnote{
	@Mik \\
	\\
	- Nachdem die Syntax analysiert wurde muss allerdings auch noch der **Inhalt** des Programmes betrachtet werden
}
\section{Semantische Analyse}

\begin{frame}{Etappen der Übersetzung: Semantische Analyse}
	\pdfpcnote{
		@Mik \\
		\\
		- Zeitlich findet die semantische Analyse nach der Syntaxanalyse statt \\
	}
	\begin{figure}[h]
		\begin{adjustbox}{max totalsize={\textwidth}{!},center}
			\begin{tikzpicture}[node distance=3mm and 1cm, inner sep=3mm]
				\node (syntactic_analysis_text) [inner sep=0] {Syntaxanalyse};
				\node (lexical_analysis) [rec, below=of syntactic_analysis_text] {Lexikalische Analyse};
				\node (syntactic_analysis) [rec, fit={(syntactic_analysis_text) (lexical_analysis)}] {};
				\node (semantic_analysis) [rec, fill=mLightBrown!35, right=of syntactic_analysis] {Semantische Analyse};
				\draw [arrow] (syntactic_analysis) -- (semantic_analysis);
				\node (codegen) [rec, right=of semantic_analysis] {Code-Erzeugung};
				\draw [arrow] (semantic_analysis) -- (codegen);
			\end{tikzpicture}
		\end{adjustbox}
	\end{figure}
\end{frame}

\begin{frame}{Semantische Analyse und Semantikregeln}
	\pdfpcnote{
		@Mik \\
		\\
		- Hier werden die semantischen Eigenschaften des Programmes validiert \\
		- Das heisst, der Inhalt des Programmes wird betrachtet und auf Fehler untersucht \\
		- Meistens werden die Regeln in einer natuerlichen Sprache definiert, da es noch kein gutes Regelsystem fuer diese Aufgabe gibt \\
	}
	\begin{itemize}
		\item Validiert die semantischen Eigenschaften
		\item Meistens: Definition in einer natürlichen Sprache
	\end{itemize}
\end{frame}

\begin{frame}{Anforderungen an die semantische Analyse (für rush)}
	\pdfpcnote{
		@Mik \\
		\\
		- Die Anforderungen an die Semantische Analyse fuer rush sind... \\
		\\
		--- \\
		\\
		1. Die Unterscheidung zwischen validen und invaliden Programmen \\
		2. Das Liefern von hilfreichen Warnungen und Informationen \\
		3. Das Hinzufuegen von Typinformationen zum Syntaxbaum \\
		4. Die Durchfuehrung von trivialen Optimierungen der Programmstruktur \\
	}
	\begin{itemize}
		\item<1-> Unterscheidung zwischen validen und invaliden Programmen
		\item<2-> hilfreiche Warnungen und Informationen
		\item<3-> Hinzufügen von Typinformationen zu dem Syntaxbaum
		\item<4-> Triviale Optimierungen der Programmstruktur
	\end{itemize}
\end{frame}

\begin{frame}{Beispiel 1: Typkonflikt}
	\pdfpcnote{
		@Mik \\
		\\
		Nun betrachten Wir einige Beispiele, die die Semantikregeln von rush verdeutlichen \\
		\\
		--- \\
		\\
		- Hier ist ein rush-Beispielprogramm dargestellt \\
		- Die Syntax des Programmes ist fehlerfrei \\
		- Allerdings enthaelt das Programm einen Semantikfehler \\
		\\
		--- \\
		\\
		- Die Variable **num** ist eine Fliesskommazahl \\
		- Der Nutzer versucht, eine Addition zwischen einer Fliesskommazahl und einer Ganzzahl durchzufuehren \\
		\\
		--- \\
		\\
		- Der von rush generierte Fehler ist unten abgebildet \\
	}

	\begin{minipage}{.5\textwidth}
		\Lirsting[float=H, fancyvrb={frame=none}]{listings/incompatible_types.rush}
	\end{minipage}%
	\begin{minipage}{.5\textwidth}
		\Darrow{Fehlerausgabe}
	\end{minipage}
	\Lirsting[float=H, fancyvrb={frame=none, fontsize=\footnotesize}, ansi=true]{listings/generated/incompatible_types.rush.out}
\end{frame}

\begin{frame}{Beispiel 2: Warnung aufgrund einer unbenutzten Variable}
	\pdfpcnote{
		@Mik \\
		\\
		- Hier ist ein weiteres Programm abgebildet \\
		- Es ist jedoch in Hinblick auf die Syntax und die Semantik fehlerfrei \\
		- Dennoch findet die semanische Analyse von rush einige Maengel in der Programmstruktur \\
		\\
		--- \\
		\\
		- Die Variable **x** ist definiert, aber nicht verwendet. \\
		- Dies erzeugt in diesem Fall eine **Warnung**, da Speicher verschwendet wird \\
		\\
		--- \\
		\\
		- Die Variable **y** wird zwar verwendet, aber ihr Wert aendert sich nie \\
		- Sie ist aber mittels des **mut** Keywoards als veraenderbar definiert \\
		- Dies erzeugt hier eine Information \\
		\\
		--- \\
		\\
		- Alle Meldungen (sowohl fuer Fehler, als auch Warnungen) enthalten den Ort der Fehlerquelle oder Warnung \\
		- Somit sind diese einfach zu lesen und Hilfreich waehrend der Entwicklung \\
	}

	\begin{minipage}{.5\textwidth}
		\Lirsting[float=H, fancyvrb={frame=none}]{listings/unused_var.rush}
	\end{minipage}%
	\begin{minipage}{.5\textwidth}
		\Darrow{Ausgabe}
	\end{minipage}
	\Lirsting[float=H, fancyvrb={frame=none, fontsize=\footnotesize}, ansi=true]{listings/generated/unused_var.rush.out}
\end{frame}

\begin{frame}{Hinzufügen von Informationen über Datentypen}
	\pdfpcnote{
		@Mik \\
		\\
		- Wie bereits erwaehnt bestehen die Aufgaben der Semantischen Analyse nicht nur aus der Ueberpruefung, sondern auch aus der Veraenderung des Syntaxbaumes \\
		- Hier sind 2 Syntaxbaueme des Programmes **1 + 42 - n** dargestellt \\
		- Der linke ist vom Parser generiert \\
		- Der rechte entsteht nach der semantischen Analyse \\
		\\
		--- \\
		\\
		- Wie bereits angesprochen werden Typinformationen zum Syntaxbaum hinzugefuegt \\
		- Hier ist dies auf der rechten Seite zu erkennen \\
		\\
		---\\
		\\
		- Zusaetzlich wird der Ausdruck **1 + 42** direkt zu **43** umgewandelt \\
		- Dies erspart somit das Berechnen des Wertes waehrend der Laufzeit, wodurch das Programm etwas schneller und kleiner wird \\
	}
	\begin{figure}[h]
		\begin{adjustbox}{max totalsize={\textwidth}{!},center}
			\begin{tikzpicture}[
					tlabel/.style={pos=0.4,right=-1pt,font=\footnotesize\color{red!70!black}},
				]
				\node(left){\Verb{1 + 42 - n}}
				child { node { \Verb{1 + 42} }
						child { node { \Verb{1} } }
						child { node { \Verb{+} } }
						child { node { \Verb{42} } }
					}
				child { node  { \Verb{-} }       }
				child { node(leftm)  { \Verb{n} } };

				\node(right)[right of=left, xshift=7cm]{\Verb{43 - n}: \emph{int}}
				child { node(rightm) { \Verb{43}: \emph{int} } }
				child { node  { \Verb{-} } }
				child { node  { \Verb{n}: \emph{int} } };

				\draw[arrow, shorten >= 0.5cm, shorten <= 0.5cm, very thick] (leftm) -- node[above] {Semantische} node[below] {Analyse} ++ (rightm);
			\end{tikzpicture}
		\end{adjustbox}
	\end{figure}

	\centering
	\LirstInline{rush}{1 + 42 - n}
\end{frame}
