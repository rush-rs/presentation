% LTeX: language=de-DE
\section{Semantische Analyse}

\begin{frame}{Etappen der Übersetzung: Semantische Analyse}
	\begin{figure}[h]
		\begin{adjustbox}{max totalsize={\textwidth}{!},center}
			\begin{tikzpicture}[node distance=1cm, inner sep=3mm]
				\node (lexical_analysis) [rec, minimum height=1.5cm] {Lexikalische Analyse};
				\node (syntactic_analysis) [rec, right=of lexical_analysis, align=center, minimum height=1.5cm] {Syntaxanalyse};
				\draw [arrow] (lexical_analysis) -- (syntactic_analysis);
				\node (semantic_analysis) [rec, right=of syntactic_analysis, align=center, minimum height=1.5cm, fill=gray!25] {Semantische\\Analyse};
				\draw [arrow] (syntactic_analysis) -- (semantic_analysis);
				\node (codegen) [rec, right=of semantic_analysis, minimum height=1.5cm] {Code-Erzeugung};
				\draw [arrow] (semantic_analysis) -- (codegen);
			\end{tikzpicture}
		\end{adjustbox}
		\caption{Etappen der Übersetzung: Semantische Analyse.}{\scite[S.~6--7]{wirth_compiler_construction_2005}}\label{fig:compilation_steps}
	\end{figure}
\end{frame}

\begin{frame}{Semantische Analyse}
	\begin{itemize}
		\item Findet nach der Syntaxanalyse und vor der Übersetzung statt
		\item Validierung der semantischen Eigenschaften des Programms
		\item Die semantischen Regeln einer Programmiersprache werden oft mittels einer natürlichen Sprache beschrieben
		\item Für rush wurde ein Dokument erstellt, welches die meisten Semantikregeln erklärt
	\end{itemize}
\end{frame}

\begin{frame}{Beispiele für die Semantikregeln von rush}
	\begin{itemize}
		\item Jede Variable, jede Funktion und jeder Parameter besitzt einen Datentyp, der nach Definierung nicht mehr geändert werden kann
		\item Eine Funktion muss immer mit den Argumenten aufgerufen werden, die zu den Parametern passen
		\item Jeder Funktionsname muss eindeutig sein
		\item Die \qVerb{main} Funktion liefert immer den \qVerb{()} Datentyp und akzeptiert keine Parameter
		\item Jede Variable muss Definiert sein, bevor diese Verwendbar ist
		\item Logische und mathematische Operationen erfolgen nur, wenn die Operanten den selben Datentypen besitzen
		\item[\ldots]
	\end{itemize}
\end{frame}

\begin{frame}{Beispiel: Invalides rush Programm}
    \begin{minipage}{.5\textwidth}
    \Lirsting[float=H, fancyvrb={frame=none}]{listings/incompatible_types.rush}
    \end{minipage}%
    \begin{minipage}{.5\textwidth}
    \Darrow{Fehlerausgabe}
    \end{minipage}
    \Lirsting[float=H, fancyvrb={frame=none, fontsize=\footnotesize}, ansi=true]{listings/generated/incompatible_types.rush.out}
\end{frame}

\begin{frame}{Beispiel 2: Invalides rush Programm}
    \begin{minipage}{.5\textwidth}
    \Lirsting[float=H, fancyvrb={frame=none}]{listings/invalid_main_fn.rush}
    \end{minipage}%
    \begin{minipage}{.5\textwidth}
    \Darrow{Fehlerausgabe}
    \end{minipage}
    \Lirsting[float=H, fancyvrb={frame=none, fontsize=\footnotesize}, ansi=true]{listings/generated/invalid_main_fn.rush.out}
\end{frame}
