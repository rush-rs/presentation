% LTeX: language=de-DE
\section{Semantische Analyse}

\begin{frame}{Etappen der Übersetzung: Semantische Analyse}
	\begin{figure}[h]
		\begin{adjustbox}{max totalsize={\textwidth}{!},center}
			\begin{tikzpicture}[node distance=3mm and 1cm, inner sep=3mm]
				\node (syntactic_analysis_text) [inner sep=0] {Syntaxanalyse};
				\node (lexical_analysis) [rec, below=of syntactic_analysis_text] {Lexikalische Analyse};
				\node (syntactic_analysis) [rec, fit={(syntactic_analysis_text) (lexical_analysis)}] {};
				\node (semantic_analysis) [rec, fill=mLightBrown!35, right=of syntactic_analysis] {Semantische Analyse};
				\draw [arrow] (syntactic_analysis) -- (semantic_analysis);
				\node (codegen) [rec, right=of semantic_analysis] {Code-Erzeugung};
				\draw [arrow] (semantic_analysis) -- (codegen);
			\end{tikzpicture}
		\end{adjustbox}
	\end{figure}
\end{frame}

\begin{frame}{Semantische Analyse und Semantikregeln}
	\begin{itemize}
		\item Validiert die semantischen Eigenschaften
		\item Meistens: Definition in einer natürlichen Sprache
	\end{itemize}
\end{frame}

\begin{frame}{Beispiel 1: Typkonflikt}
	\begin{minipage}{.5\textwidth}
		\Lirsting[float=H, fancyvrb={frame=none}]{listings/incompatible_types.rush}
	\end{minipage}%
	\begin{minipage}{.5\textwidth}
		\Darrow{Fehlerausgabe}
	\end{minipage}
	\Lirsting[float=H, fancyvrb={frame=none, fontsize=\footnotesize}, ansi=true]{listings/generated/incompatible_types.rush.out}
\end{frame}

\begin{frame}{Beispiel 2: Warnung aufgrund einer unbenutzten Variable}
	\begin{minipage}{.5\textwidth}
		\Lirsting[float=H, fancyvrb={frame=none}]{listings/unused_var.rush}
	\end{minipage}%
	\begin{minipage}{.5\textwidth}
		\Darrow{Ausgabe}
	\end{minipage}
	\Lirsting[float=H, fancyvrb={frame=none, fontsize=\footnotesize}, ansi=true]{listings/generated/unused_var.rush.out}
\end{frame}

\begin{frame}{Anforderungen an die semantische Analyse (für rush)}
	\begin{itemize}
		\item<1-> Unterscheidung zwischen validen und invaliden Programmen
		\item<2-> Hilfreiche Warnungen und Informationen
		\item<3-> Hinzufügen von Typinformationen zu dem AST
		\item<4-> Triviale Optimierungen der Programmstruktur
	\end{itemize}
\end{frame}

\begin{frame}{Hinzufügen von Informationen über Datentypen}
	\begin{figure}[h]
		\begin{adjustbox}{max totalsize={\textwidth}{!},center}
			\begin{tikzpicture}[
					tlabel/.style={pos=0.4,right=-1pt,font=\footnotesize\color{red!70!black}},
				]
				\node(left){\Verb{1 + 42 - n}}
				child { node { \Verb{1 + 42} }
						child { node { \Verb{1} } }
						child { node { \Verb{+} } }
						child { node { \Verb{42} } }
					}
				child { node  { \Verb{-} }       }
				child { node(leftm)  { \Verb{n} } };

				\node(right)[right of=left, xshift=7cm]{\Verb{43 - n}: \emph{int}}
				child { node(rightm) { \Verb{43}: \emph{int} } }
				child { node  { \Verb{-} } }
				child { node  { \Verb{n}: \emph{int} } };

				\draw[arrow, shorten >= 0.4cm, shorten <= 0.4cm, very thick] (leftm) -- node[above] {Semantische} node[below] {Analyse} ++ (rightm);
			\end{tikzpicture}
		\end{adjustbox}
	\end{figure}
\end{frame}
