% LTeX: language=de-DE
% chktex-file -2
\section{Kompilierung zu \riscv}
\begin{frame}{Was ist \riscv?}
	\pdfpcnote{
		@Mik \\
		\\
		1. **RISC** -> **R**educed **I**nstruction **Set** **C**omputer (wenige Befehlscodes) \\
		2. Forschungsprojekt der UC Berkeley \\
		3. Loesen der Probleme anderer Architekturen \\
		4. Kernwerte sind: **Simplizitaet und Erweiterbarkeit** \\
		5. Google, Microsoft, Samsung, IMB \\
	}

	\begin{itemize}
		\item<1-> \textbf{R}educed \textbf{I}nstruction \textbf{S}et \textbf{C}omputer (\cemph{RISC})
		\item<2-> \cemph{Forschungsprojekt} der UC Berkeley
		\item<3-> Lösen der Probleme vieler anderer Architekturen
		\item<4-> \cemph{Simplizität} und Erweiterbarkeit
		\item<5-> Unterstüzung durch: Google, Microsoft, Samsung und IBM
	\end{itemize}
\end{frame}

\begin{frame}{Beispiel}
	\pdfpcnote{
		@Mik \\
		\\
		- Compiler generiert Assembly \\
		- Alle Eigenschaften muessen beruecksichtigt werden \\
		- Selbst einfache Operationen sind komplex (**Prologue**) \\
		- LLVM **call-Anweisung**, das geht hier leider nicht mehr \\
	}
	\begin{minipage}{0.4\textwidth}
		\vspace{.5cm}
		\Lirsting[float=H, fancyvrb={frame=none, fontsize=\small}]{deps/paper/listings/fib.rush}
		\centering
		\vspace{-.5cm}
		\Larrow{Ausgabe}
	\end{minipage}
	\hfill
	\begin{minipage}{0.4\textwidth}
		\Lirsting[float=H, fancyvrb={frame=none, fontsize=\scriptsize}, ranges={1-13, 28-40}]{listings/generated/fib_riscv.s}
	\end{minipage}
\end{frame}

\begin{frame}{Vorteile}
	\pdfpcnote{
		@Mik \\
		\\
		- Waehrend der Entwicklung dieses Compilers haben sich folgende Aspekte als vorteilhaft herausgestellt: \\
		- Die Architektur ist sehr neu und modern \\
		- Sie ist komplett open-source und gemeinschaftlich verwaltet \\
		- Zudem ist die Architektur sehr uebersichtlich und simel \\
		- Es gibt zudem sehr gute und uebersichtliche offizielle Dokumentation \\
	}

	\begin{itemize}
		\item<1-> Sehr neu und modern
		\item<2-> Komplett open-source und gemeinschaftlich verwaltet
		\item<3-> Sehr übersichtliche und simple Architektur
		\item<4-> Sehr gute und übersichtliche Dokumentation             \end{itemize}
\end{frame}

\begin{frame}{Nachteile}
	\pdfpcnote{
		@Mik \\
		\\
		- Geringer Verbreitung -> schwierige Google Suche \\
		- Operationen, wie z.B. **Funktionsaufrufe**, sind aufwendiger \\
        - Ausfuehren und Debugging; Abhaengig von einem Emulator (QEMU) \\
		\\
		--- \\
		\\
		- Dennoch habe ich die Arbeit mit der Architektur als sehr angenehm und zukunftssicher empfunden \\
	}

	\begin{itemize}
		\item<1-> Geringe Verbreitung
		\item<2-> Weniger Online-Ressourcen
		\item<3-> Einige Operationen sind aufwendiger
		\item<4-> Abhängigkeit von einem Emulator (QEMU)
	\end{itemize}
\end{frame}
