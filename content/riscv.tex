% LTeX: language=de-DE
% chktex-file -2
\section{Kompilierung zu \riscv}
\begin{frame}{Was ist \riscv?}
	\pdfpcnote{
		- RISC -> Reduced Instruction Set Computer
		- Forschungsprojekt der UC Berkeley
		- Lösen der Probleme von anderen Architekturen
		- Simplizit und Erweiterbarkeit
		- Unterstützt durch
		- Google
		- Microsoft
		- Samsung
		- IBM
	}

	\begin{itemize}
		\item<1-> \textbf{R}educed \textbf{I}nstruction \textbf{S}et \textbf{C}omputer (\cemph{RISC})
		\item<2-> \cemph{Forschungsprojekt} der UC Berkeley
		\item<3-> Lösen der Probleme vieler anderer Architekturen
		\item<4-> \cemph{Simplizität} und Erweiterbarkeit
		\item<5-> Unterstüztung durch: Google, Microsoft, Samsung und IBM
	\end{itemize}
\end{frame}

\begin{frame}{Beispiel}
	\begin{minipage}{0.4\textwidth}
		\vspace{.5cm}
		\Lirsting[float=H, fancyvrb={frame=none, fontsize=\small}]{deps/paper/listings/fib.rush}
		\centering
		\vspace{-.5cm}
		\Larrow{Ausgabe}
	\end{minipage}
	\hfill
	\begin{minipage}{0.4\textwidth}
		\Lirsting[float=H, fancyvrb={frame=none, fontsize=\scriptsize}, ranges={1-13, 28-40}]{listings/generated/fib_riscv.s}
	\end{minipage}
\end{frame}

\begin{frame}{Fazit zu \riscv}
	\begin{table}[h]
		\rowcolors{2}{gray!15}{}
		\begin{tabularx}{0.95\textwidth}{L|L}
			\cellcolor{green!20} Vorteile                       & \cellcolor{red!20} Nachteile        \\
			\hline
			Sehr neu und modern                                 & Geringe Verbreitung                 \\
			Komplett open-source und Gemeinschaftlich verwaltet & Eher experimentell                  \\
			Sehr übersichtliche und simple Architektur          & Einige Operationen sind aufwendiger \\
			Sehr gute und übersichtliche Dokumentation          & Weniger Online-Ressourcen           \\
		\end{tabularx}
	\end{table}
\end{frame}
