% LTeX: language=de-DE
% chktex-file -2
\section{Kompilierung zu \riscv}
\begin{frame}{Was ist \riscv?}
	\pdfpcnote{
		@Mik \\
		\\
		- **RISC** steht fuer **R**educed **I**nstruction **Set** **C**omputer, da es eher wenige Befehlscodes besitzt \\
		- Die Architektur entstand im Ramen eines Forschungsprojektes der UC Berkeley \\
		- Die Architektur versucht, die Probleme anderer Architekturen zu loesen \\
		- Kernwerte sind hierbei: **Simplizitaet und Erweiterbarkeit** \\
		- Heutzutage wird die Organisation durch... \\
            - Google \\
            - Microsoft \\
            - Samsung \\
            - und IBM \\
        - ...unterstuetzt \\
	}

	\begin{itemize}
		\item<1-> \textbf{R}educed \textbf{I}nstruction \textbf{S}et \textbf{C}omputer (\cemph{RISC})
		\item<2-> \cemph{Forschungsprojekt} der UC Berkeley
		\item<3-> Lösen der Probleme vieler anderer Architekturen
		\item<4-> \cemph{Simplizität} und Erweiterbarkeit
		\item<5-> Unterstüzung durch: Google, Microsoft, Samsung und IBM
	\end{itemize}
\end{frame}

\begin{frame}{Beispiel}
    \pdfpcnote{
        @Mik \\
        \\
        - Hier ist ein Teil der Ausgabe des Fibonaccibeispiels dargestellt \\
        - Dieser Compiler generiert Assembly Code fuehr die RISC-V Architektur \\
        - Hierbei muessen alle Eigenschaften der Architektur, wie Register und Speicher, beachtet werden \\
        - Besonders ist, selbst die Einfachsten Operationen, wie **Funktionsaufrufe** benoetigen hier zahlreiche Anweisungen \\
        - Bei WebAssembly und LLVM konnte einfach eine **call-Anweisung** verwendet werden, das geht hier leider nicht mehr \\
    }
	\begin{minipage}{0.4\textwidth}
		\vspace{.5cm}
		\Lirsting[float=H, fancyvrb={frame=none, fontsize=\small}]{deps/paper/listings/fib.rush}
		\centering
		\vspace{-.5cm}
		\Larrow{Ausgabe}
	\end{minipage}
	\hfill
	\begin{minipage}{0.4\textwidth}
		\Lirsting[float=H, fancyvrb={frame=none, fontsize=\scriptsize}, ranges={1-13, 28-40}]{listings/generated/fib_riscv.s}
	\end{minipage}
\end{frame}

\begin{frame}{Fazit zu \riscv}
    \pdfpcnote{
        @Mik \\
        \\
        - Waehrend der Entwicklung dieses Compilers haben sich folgende Aspekte als vorteilhaft herausgestellt: \\
            - Die Architektur ist sehr neu und modern \\
            - Sie ist komplett open-source und gemeinschaftlich verwaltet \\
            - Zudem ist die Architektur sehr uebersichtlich und simel \\
            - Es gibt zudem sehr gute und uebersichtliche offizielle Dokumentation \\
        \\
        --- \\
        \\
        - Allerdings haben sich folgende Aspekte als Nachteilhaft herausgestellt: \\
            - Die geringe Verbreitung der Architektur erschwert eigene Google-Anfragen \\
            - Einige Operationen, wie z.B. **Funktionsaufrufe** sind aufwendiger \\
            - Zudem ist man beim Ausfuehren und Debugging auf einen Emulator, in unserem Fall **QEMU** angewiesen \\
        \\
        --- \\
        \\
        - Dennoch habe ich die Arbeit mit der Architektur als sehr angenehm und zukunftssicher empfunden \\
    }
	\begin{table}[h]
		\rowcolors{2}{gray!15}{}
		\begin{tabularx}{0.95\textwidth}{L|L}
			\cellcolor{green!20} Vorteile                       & \cellcolor{red!20} Nachteile        \\
			\hline
			Sehr neu und modern                                 & Geringe Verbreitung                 \\
			Komplett open-source und gemeinschaftlich verwaltet & Weniger Online-Ressourcen           \\
			Sehr übersichtliche und simple Architektur          & Einige Operationen sind aufwendiger \\
            Sehr gute und übersichtliche Dokumentation          & Abhaengigkeit von einem Emulator (QEMU)  \\
		\end{tabularx}
	\end{table}
\end{frame}
