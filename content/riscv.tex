% LTeX: language=de-DE
% chktex-file -2
\section{Kompilierung zu \riscv}
\begin{frame}{Was ist \riscv?}
	\begin{itemize}
		\item \textbf{R}educed \textbf{I}nstruction \textbf{S}et \textbf{C}omputer (RISC)
		\item Forschungsprojekt der UC Berkeley
		\item Ziel: Lösen der Probleme vieler CISC Architekturen
		\item Simplizität und Erweiterbarkeit
		\item Unterstüztung durch: Google Microsoft, Samsung und IBM
	\end{itemize}
\end{frame}

\begin{frame}{Register der \riscv{} Architektur}
	\begin{table}
		\centering
		\begin{tabularx}{\linewidth}{L|L}
			\rowcolor{gray!15} Register & Verwendung                        \\ \hline
			\texttt{zero}               & hardwired zero                    \\ \hline
			\texttt{ra}                 & return address                    \\ \hline
			\texttt{sp}                 & stack pointer                     \\ \hline
			\texttt{t0}--\texttt{t6}    & temporary storage                 \\ \hline
			\texttt{fp}                 & frame pointer                     \\ \hline
			\texttt{a0}, \texttt{a1}    & function arguments, return values \\ \hline
			\texttt{a2}--\texttt{a7}    & function arguments                \\ \hline
			\texttt{s1}--\texttt{s11}   & saved register                    \\ \hline
			\texttt{fa0}, \texttt{fa1}  & float arguments, return values    \\ \hline
			\texttt{fa2}--\texttt{fa7}  & float arguments                   \\ \hline
			\texttt{fs0}--\texttt{fs11} & float saved registers             \\ \hline
			\texttt{ft0}--\texttt{ft11} & float temporaries                 \\
		\end{tabularx}
	\end{table}
\end{frame}

\begin{frame}{Beispiel}
	\begin{minipage}{0.35\textwidth}
		\TODO{maybe fib?}
		\Lirsting[float=H, fancyvrb={frame=none}]{deps/paper/listings/simple_add.rush}
		\centering
		\Larrow{Ausgabe}
	\end{minipage}
	\hfill
	\begin{minipage}{0.45\textwidth}
		\Lirsting[float=H, fancyvrb={frame=none, fontsize=\scriptsize}]{deps/paper/listings/generated/rush_simple_add.s}
	\end{minipage}
\end{frame}

\begin{frame}{Fazit zu \riscv}
	\begin{table}[h]
		\rowcolors{2}{gray!15}{}
		\begin{tabularx}{0.95\textwidth}{L|L}
			\cellcolor{green!20} Vorteile                       & \cellcolor{red!20} Nachteile               \\
			Sehr neu und modern                                 & Geringe Verbreitung                        \\
			Komplett open-source und Gemeinschaftlich verwaltet & Eher experimentell                         \\
			Sehr übersichtliche und simple Architektur          & Einige Operationen sind aufwendiger        \\
			Weniger Online-Ressourcen                           & Sehr gute und übersichtliche Dokumentation \\
		\end{tabularx}
	\end{table}
\end{frame}
