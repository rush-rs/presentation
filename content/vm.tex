% LTeX: language=de-DE
\section{Virtuelle Maschine}

\begin{frame}{Virtuelle Maschine}
	\begin{itemize}
		\item<1-> Häufig: Eine \emph{Virtuelle Maschine} (VM) simuliert echte Computer
		      \begin{itemize}
			      \item<2->  Display
			      \item<3-> Lautsprecher
			      \item<4-> Festplatte
			      \item<4-> \dots
		      \end{itemize}
		\item<5-> Hier: Software, die wie die CPU eines Rechners funktioniert
	\end{itemize}
\end{frame}

\begin{frame}{Übertragung der Konzepte auf die rush VM}
	\begin{center}
		\begin{minipage}{.5\textwidth}
			\Lirsting[fancyvrb={frame=none}, float=H]{listings/vm_struct.rs}
		\end{minipage}
	\end{center}
	\begin{description}
		\item<1->[stack] für temporäre Werte
		\item<2->[mem] für Variablen
		\item<3->[mem\_ptr] Index der letzten freien Speicherzelle
        \item<4->[call\_stack] Aufrufstapel (\emph{Befehlsähler} und \emph{Funktionszähler})
	\end{description}
\end{frame}

\begin{frame}{Struktur der Programme der rush VM}
	\begin{itemize}
		\item<1-> Unterteilung in Funktionen
		      \begin{itemize}
			      \item<2-> Ohne Namen
			      \item<3-> numerische Identifizierung
			      \item<4-> Enthält mehrere Anweisungen
		      \end{itemize}
		\item<5-> ca. 30 verschiedene Befehlscodes
		\item<6-> Struktur der Anweisungen: \enquote{\LirstInline{asm}{call 2}}
		      \begin{itemize}
			      \item<7-> Befehlscode (\texttt{call})
			      \item<8-> Optionaler Operand (\texttt{2})
		      \end{itemize}
	\end{itemize}
\end{frame}

\begin{frame}{Demonstration: Ein-/Ausgabe}
	\begin{minipage}{0.5\textwidth}
		\Lirsting[float=H, fancyvrb={frame=none, fontsize=\small}]{deps/paper/listings/fib.rush}
		\centering
		\Larrow{Ausgabe}
	\end{minipage}
	\hfill
	\begin{minipage}{0.35\textwidth}
		\Lirsting[float=H, fancyvrb={frame=none, fontsize=\footnotesize}, ranges={1-20,26-32}]{listings/vm_fib.s}
	\end{minipage}
\end{frame}

\begin{frame}{Demonstration: Laufzeitverhalten}
	\begin{figure}[H]
		\href{run:assets/01_rush_presentation_vm.mkv}{
			\movie{\includegraphics[width=.95\textwidth]{assets/01_rush_presentation_vm.png}}{assets/01_rush_presentation_vm.mkv}
		}
	\end{figure}
\end{frame}

\begin{frame}{VM: Fazit}
	\begin{itemize}
		\item<1-> Ca. 2.7 mal schneller als der Tree-walking Interpreter
		\item<2-> Einfache Implementierung des Compilers
		      \begin{itemize}
			      \item<3-> Stack-basierte Architektur
                  \item<4-> Gleichzeitige Entwicklung von VM und Compiler (\cemph{Feedbackschleife})
			      \item<5-> Hoher Abstraktionsgrad
		      \end{itemize}
	\end{itemize}
\end{frame}
