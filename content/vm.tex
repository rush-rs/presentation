% LTeX: language=de-DE
\section{Virtuelle Maschinen}

\begin{frame}{Virtuelle Maschinen}
	\begin{itemize}
		\item Oft bezeichnet eine \emph{virtuelle Maschine} (VM) ein Softwareprogramm, welches einen echten Computer simuliert
		\item Hierbei werden oft auch Geräte wie das Display, Lautsprecher oder die Festplatte miteinbezogen
		\item In diesem Kontext bezeichnet der Begriff jedoch eine Software, die wie die CPU eines Rechners funktioniert
	\end{itemize}
\end{frame}

\begin{frame}{Wie eine CPU Programme ausführt}
	\begin{itemize}
		\item Die meisten Prozessoren basieren auf der \emph{von Neumann Architektur}~\scite[p.~172]{Ledin2020-yp}
		\item Eine CPU enthält nach von Neumann ein \emph{Rechenwerk}\footnote{Engl: \enquote{arithmetic logic unit} (ALU).}, \emph{Steuerwerk}\footnote{Engl: \enquote{control unit}.}, \emph{Speicherwerk}, \emph{Ein- / Ausgabewerk} und ein Bussystem~\scite[p.~172]{Ledin2020-yp}
		\item Die Programmausführung wird durch den sog. \emph{Befehlszylus}\footnote{Engl: \enquote{fetch-decode-execute cycle}.} modelliert~\scite[pp.~208-209]{Ledin2020-yp}:
		\item[] \begin{enumerate}
				\item \textbf{Fetch} (Befehl laden): Das Steuerwerk lädt die nächste Anweisung aus dem Speicher
				\item \textbf{Decode} (Befehl dekodieren): Der Befehlscode und die Operanden werden ermittelt
				\item \textbf{Execute} (Befehl ausführen): Die zuständige Einheit im Prozessor wird verwendet, um den Befehl zu verarbeiten.
				      Beispielsweis wird das Rechenwerk für logische und mathemtische Befehle aufgerufen.
			\end{enumerate}
	\end{itemize}
\end{frame}

\begin{frame}{Übertragung der Konzepte auf die rush VM}
	\Lirsting[ranges={16-26}, caption={Struct Definition der VM.}, label={lst:vm_struct}, fancyvrb={fontsize=\footnotesize}, float=H]{deps/rush/crates/rush-interpreter-vm/src/vm.rs}
	\begin{itemize}
		\item \TODO{fix broken caption}
		\item \qVerb{stack}: Speicher für temporäre Werte bei komplexeren Operationen
		\item \qVerb{mem}: Anhaltender Speicher mit einer festen Größe für Variablen
		\item \qVerb{mem_ptr}: Hält den Index der letzten freien Speicherzelle in \qVerb{mem}
		\item \qVerb{call_stack}: Aufrufstapel, welcher den \emph{Befehlsähler} und den \emph{Funktionszähler} für jeden Aufruf speichert
	\end{itemize}
\end{frame}

\begin{frame}{Speicherstruktur der rush VM.}
	\begin{itemize}
		\item Unterscheidung zwischen zwei Arten der Adressierung
		\item \emph{relative Adressierung}: \qVerb{svari *rel[0]}
		\item \emph{absolute Adressierung}: \qVerb{svari *abs[0]}
	\end{itemize}

	\begin{figure}
		\centering
		\begin{NiceTabular}{>{\scriptsize}c}[name=Left]
			\\
			\\
			\\
			num              \\
			\Block[draw]{} 9 \\
		\end{NiceTabular}\hspace{1cm}
		\begin{NiceTabular}
			[
				first-col,
				%code-for-first-col=\ValueMinusOne{iRow},
				first-row,
				hvlines,
				colortbl-like,
				name = Right
			]
			{cc>{\scriptsize}c}
			 & cell                  & rel    & {\normalsize abs} \\
			 & a                     & $-3$   & $mp + rel = 0$    \\
			 & b                     & $-2$   & $mp + (-2) = 1$   \\
			 & c                     & $-1$   & $mp - 1 = 2$      \\
			 & d                     & $0$    & $3$               \\
			 & \cellcolor{gray!30} e & $1$    & $mp + 1 = 4$      \\
			 & \ldots                & \ldots & \ldots            \\
		\end{NiceTabular}

		\begin{tikzpicture}[overlay,remember picture]
			\draw [->] (Left-5.5-|Left-last) to [bend left] (Right-5.5-|Right-1);
			\draw [thick, dashed] (Right-4.5-|Right-5) --  node[anchor=west, xshift=-1cm, align=center] {\scriptsize memory\\ \scriptsize pointer $= 3$} ([xshift=2.5cm]Right-4.5-|Right-4);
		\end{tikzpicture}
		\caption{Speicherstruktur der rush VM.}\label{fig:rush_vm_linmem}
	\end{figure}
\end{frame}
