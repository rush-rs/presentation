\section{Finale Anmerkungen \& Fazit}
\begin{frame}{Wir haben gelernt \dots}
	\pdfpcnote{
		@Mik \\
		\\
		1. Waehrend der Arbeit an unserer besonderen Lernleistung haben wir vertiefendes Wissen ueber... \\
            - Lexer und Parser \\
            - und Tree-walking interpreter gesammelt \\
            - Hier war besonders die Implementierung von Pointern eine Herrausvorderung \\
		\\
		--- \\
		\\
		2. Zudem haben wir den Algorithmus fuer Pratt Parsing und seine Vorteile kennengelernt \\
		3. Die Grundlagen des Compilerbaus haben wir anhand von LLVM und WebAssembly erschlossen \\
		4. Abschliessend haben wir verschiedene Assembly-Dialekte gelernt und beherrschen nun die low-level Programmierung \\
	}
	\begin{itemize}
		\item<1-> Vertiefung
			\begin{itemize}
				\item <2-> Lexer und Parser
				\item <3-> Tree-walking Interpreter
			\end{itemize}
		\item<4-> Pratt Parsing
		\item<5-> LLVM und WebAssembly
		\item<6-> Assembly und low-level Programmierung
	\end{itemize}
\end{frame}

\begin{frame}{Links}
    \pdfpcnote{
        @Mik \\
        \\
        1. Fuer eine Uebersicht ueber rush koennen Sie gerne die **rush Website** besuchen \\
        2. Zudem kann die neuste version unserer schriftlichen Arbeit unter der **paper** subdomain eingesehen werden  \\
        3. Sie koennen anhand eigener Programme die genannten rush Compiler und Interprter unter der **play** subdomain ausprobieren \\
        4. Letztlich kann der Quelltext aller 19 Teilprojekte von rush auf GitHub eingesehen und bearbeitet werden \\
        \\
        --- \\
        \\
        - Wir danken Ihnen fuehr ihre Aufmerksamkeit \\
    }
	\centering
	\begin{minipage}{.5\textwidth}
		\begin{description}
			\item<1->[rush Website] \url{https://rush-lang.de}
			\item<2->[Paper] \url{https://paper.rush-lang.de}
			\item<3->[Playground] \url{https://play.rush-lang.de}
			\item<4->[GitHub] \url{https://github.com/rush-rs}
		\end{description}
	\end{minipage}
\end{frame}
