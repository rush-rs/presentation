% LTeX: language=de-DE
% chktex-file -2
\section{Kompilierung zu LLVM}
\begin{frame}{Was ist LLVM?}
	\pdfpcnote{
		@Mik \\
		\\
		1. Startete als Forschungsprojekt \\
		2. Wird dafuer verwendet, Code aus einer Zwischendarstellung, auch **IR** genannt, zu erzeugen \\
		3. Die IR kann mittels oeffentlichen APIs erzeugt werden \\
		4. Das Framework ist bekannt fuehr seine agressiven Optimierungsmassnahmen \\
		5. Auch die Programmierspachen Rust und Swift nutzen LLVM \\
		6. LLVM ist also das Backend eines Compilers \\
	}
	\begin{itemize}
		\item<1-> Startete als \cemph{Forschungsprojekt}
		\item<2-> Erzeugung von Code aus einer \cemph{Zwischendarstellung} (IR)
		\item<3-> Aggressive \cemph{Optimierung}
		\item<4-> Die IR kann mittels APIs erzeugt werden
		\item<5-> Auch Rust und Swift nutzen LLVM
			\item[\Rightarrow]<6-> Das Backend eines Compilers
	\end{itemize}
\end{frame}

\begin{frame}{Rolle von LLVM in einem Compiler}
	\pdfpcnote{
		@Mik \\
		\\
		- Um LLVM in einen Compiler zu integrieren benoetigt man zunaechst eine Komponente, die die LLVM IR erzeugt \\
		- Anschliessend kann LLVM die IR dann verarbeite, optimieren und zur Codeerzeugung nutzen \\
	}
	\begin{figure}[h]
		\begin{adjustbox}{max totalsize={\textwidth}{!},center}
			\begin{tikzpicture}[node distance=3mm and 1cm, inner sep=3mm]
				\node (syntactic_analysis_text) [inner sep=0] {Syntaxanalyse};
				\node (lexical_analysis) [rec, below=of syntactic_analysis_text] {Lexikalische Analyse};
				\node (syntactic_analysis) [rec, fit={(syntactic_analysis_text) (lexical_analysis)}] {};

				\node (semantic_analysis) [rec, align=center, right=of syntactic_analysis] {Semantische\\Analyse};
				\draw [arrow] (syntactic_analysis) -- (semantic_analysis);

				\node (ir_generation) [rec, align=center, fill=mLightBrown!25, right=of semantic_analysis] {LLVM IR\\Generation};
				\draw [arrow] (semantic_analysis) -- (ir_generation);

				\node (llvm) [rec, align=center, fill=mLightBrown!25, right=of ir_generation] {LLVM\\Backend};
				\draw [arrow] (ir_generation) -- (llvm);
			\end{tikzpicture}
		\end{adjustbox}
	\end{figure}
\end{frame}

\begin{frame}{Der rush LLVM Compiler}
	\pdfpcnote{
		@Mik \\
		\\
		- Diese Konzepte wurden auch auf den rush LLVM Compiler uebertragen \\
		--- \\
		\\
		- Der rush Compiler erzeugt LLVM IR \\
		- Da rush in Rust geschrieben ist nutzen wir eine Rust library namens **Inkwell**, um LLVM IR zu erzeugen \\
	}
	\begin{itemize}
		\item Rust library names \cemph{Inkwell}
		\item Der Compiler erzeugt LLVM IR
	\end{itemize}
\end{frame}

\begin{frame}{Beispiel Ein-/Ausgabe}
	\pdfpcnote{
		@Mik \\
		\\
		- Nun betrachten wir erneut das Fibonaccibeispiel \\
		- Die rechte Seite Zeigt die LLVM IR, die der rush Compiler generiert hat \\
		\\
		--- \
		\\
		- Es ist zu erkennen, dass die IR einige Konstrukte mit hohem Abstraktionsgrad bereitstellt... \\
		- Wie zum Beispiel das Prinzip von Funktionen oder die **call** Anweisung \\
	}
	\begin{minipage}{0.45\textwidth}
		\Lirsting[float=H, fancyvrb={frame=none, fontsize=\small}]{deps/paper/listings/fib.rush}
		\centering
		\Larrow{Ausgabe}
	\end{minipage}
	\hfill
	\begin{minipage}{0.5\textwidth}
		\Lirsting[float=H, fancyvrb={frame=none, fontsize=\scriptsize}]{listings/fib.ll}
	\end{minipage}
\end{frame}

\begin{frame}{Fazit}
	\pdfpcnote{
		@Mik \\
		\\
		- Waehrend der Entwicklung dieses Compilers haben sich unter anderem... \\
            - Der hohe Abstraktionsgrad \\
            - die Unabhaenhigeit von der Zielmaschine \\
            - und die aggressiven Optimierungsmassnahmen \\
            - ...als vorteilhaft dargestellt \\
        - Letzere fuehren dazu, dass ein x64 Programm ca 1.7 mal schneller als eines von einem nicht-optimierenden Compiler ausgefuehrt wird \\
		\\
		--- \\
		\\
		- Allerdings haben sich auch einige Nachteile herrausgestellt: \\
            - Die Installation der LLVM libraries ist sehr aufwendig \\
            - Zudem ist der fertige Compiler sehr gross \\
            - Die Inkwell library war sehr schlecht dokumentiert, weshalb wir den Quelltext der Library betrachten mussten \\
            - Zusaetzlich ist man als Rust Entwickler abhaengig von einer C++ Codebase \\
        \\
        --- \\
        \\
        - **Dennoch wuerden wir, wenn wir einen komplexeren Compiler entwerfen wuerden, LLVM in Betracht ziehen, da es sehr weit verbreitet ist \\
	}
	\begin{table}[h]
		\rowcolors{2}{gray!15}{}
		\begin{tabularx}{0.95\textwidth}{L|L}
			\cellcolor{green!20} Vorteile                             & \cellcolor{red!20} Nachteile               \\
			\hline
			Hoher Abstraktionsgrad                                    & Aufwendige Installation der LLVM Libraries \\
			Unabhängigkeit von der Zielmaschine                       & Signifikante Größe des Compilers           \\
			Aggressive Optimierungsmaßnahmen                          & Unvollständige Dokumentation von Inkwell   \\
			Ausgabeprogramm ca. 1,7 mal schneller (vgl. x64 Compiler) & Abhängigkeit von einer C++ Codebase        \\
		\end{tabularx}
	\end{table}
\end{frame}
