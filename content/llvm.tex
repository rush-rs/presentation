% LTeX: language=de-DE
% chktex-file -2
\section{Kompilierung zu LLVM}

\begin{frame}{Was ist LLVM?}
    \pdfpcnote{
        @Mik \\
        \\
        1. Forschungsprojekt \\
        2. Codeerzeugung aus einer Zwischendarstellung \\
        3. Die IR kann mittels oeffentlichen APIs erzeugt werden \\
        4. fuehr seine agressiven Optimierungsmassnahmen \\
        5. Rust und Swift nutzen LLVM \\
        6. => **LLVM ist also das Backend eines Compilers** \\
    }
    \begin{itemize}
        \item<1-> Startete als \cemph{Forschungsprojekt}
        \item<2-> Erzeugung von Code aus einer \cemph{Zwischendarstellung} (IR)
        \item<2-> Die IR kann mittels APIs erzeugt werden
        \item<3-> Aggressive \cemph{Optimierung}
        \item<4-> Auch Rust und Swift nutzen LLVM
            \item[\Rightarrow]<6-> Das Backend eines Compilers
    \end{itemize}
\end{frame}

\begin{frame}{Rolle von LLVM in einem Compiler}
    \pdfpcnote{
        @Mik \\
        \\
        - Komponente, die die LLVM IR erzeugt \\
        - LLVM verarbeitet dann diese IR\\
        1. Pruefen
        2. Optimieren
        3. Uebersetzen
    }
    \begin{figure}[h]
        \begin{adjustbox}{max totalsize={\textwidth}{!},center}
            \begin{tikzpicture}[node distance=3mm and 1cm, inner sep=3mm]
                \node (syntactic_analysis_text) [inner sep=0] {Syntaxanalyse};
                \node (lexical_analysis) [rec, below=of syntactic_analysis_text] {Lexikalische Analyse};
                \node (syntactic_analysis) [rec, fit={(syntactic_analysis_text) (lexical_analysis)}] {};

                \node (semantic_analysis) [rec, align=center, right=of syntactic_analysis] {Semantische\\Analyse};
                \draw [arrow] (syntactic_analysis) -- (semantic_analysis);

                \node (ir_generation) [rec, align=center, fill=mLightBrown!25, right=of semantic_analysis] {LLVM IR\\Generation};
                \draw [arrow] (semantic_analysis) -- (ir_generation);

                \node (llvm) [rec, align=center, fill=mLightBrown!25, right=of ir_generation] {LLVM\\Backend};
                \draw [arrow] (ir_generation) -- (llvm);
            \end{tikzpicture}
        \end{adjustbox}
    \end{figure}
\end{frame}

\begin{frame}{Der rush LLVM Compiler}
    \pdfpcnote{
        @Mik \\
        \\
        - Diese Konzepte wurden auch auf den rush LLVM Compiler uebertragen \\
        - Der rush Compiler erzeugt LLVM IR \\
        - Rust library namens **Inkwell**, um LLVM IR zu erzeugen \\
    }
    \begin{itemize}
        \item Rust library names \cemph{Inkwell}
        \item Der Compiler erzeugt LLVM IR
    \end{itemize}
\end{frame}

\begin{frame}{Beispiel Ein-/Ausgabe}
    \pdfpcnote{
        @Mik \\
        \\
        - Nun betrachten wir erneut das Fibonaccibeispiel \\
        - Rechts: LLVM IR, die der rush Compiler generiert hat \\
        \\
        --- \
        \\
        - einige Konstrukte mit hohem Abstraktionsgrad \\
        - die **call** Anweisung \\
        - If-Konstrukte sind allerdings schwieriger \\
    }
    \begin{minipage}{0.45\textwidth}
        \Lirsting[float=H, fancyvrb={frame=none, fontsize=\small}]{deps/paper/listings/fib.rush}
        \centering
        \Larrow{Ausgabe}
    \end{minipage}
    \hfill
    \begin{minipage}{0.5\textwidth}
        \Lirsting[float=H, fancyvrb={frame=none, fontsize=\scriptsize}]{listings/fib.ll}
    \end{minipage}
\end{frame}

\begin{frame}{Vorteile}
    \pdfpcnote{
        @Mik \\
        \\
        1. Der hohe Abstraktionsgrad \\
        2. die Unabhaenhigeit von der Zielmaschine \\
        3. und die aggressiven Optimierungsmassnahmen \\
        4. x64 Programm 1.7 mal schneller als nicht-optimiert \\
    }

    \begin{itemize}
        \item<1->  Hoher Abstraktionsgrad
        \item<2-> Unabhängigkeit von der Zielmaschine
        \item<3-> Aggressive Optimierungsmaßnahmen
        \item<4-> Ausgabeprogramm ca. 1,7 mal schneller (vgl. x64 Compiler)
    \end{itemize}
\end{frame}

\begin{frame}{Nachteile}
    \pdfpcnote{
        @Mik \\
        \\
        1. Installation aufwendig \\
        2. der fertige Compiler sehr gross \\
        3. Inkwell schlecht dokumentiert \\
        4. Abhaengigkeit von einer C++ Codebase \\
        \\
        --- \\
        \\
        - **Wir wuerden LLVM nochmal verwenden** \\
    }
    \begin{itemize}
        \item<1-> Aufwendige Installation der LLVM libraries
        \item<2-> Signifikante Größe des Compilers
        \item<3-> Unvollständige Dokumentation von Inkwell
        \item<4-> Abhängigkeit von einer C++ Codebase
    \end{itemize}
\end{frame}
