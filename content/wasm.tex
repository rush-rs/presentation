% chktex-file -2
% LTeX: language=de-DE
\section{Kompilierung zu WebAssembly}
\begin{frame}{Was ist WebAssembly?}
	\pdfpcnote{
		@Silas \\
		\\
		1. WebAssembly ist ein sicheres, portables, kompaktes und effizientes Format gedacht fuer leistungsstarke Webanwendungen \\
		2. grundsaetzlich nur eine alleinstehende Spezifikation des Formats \\
		3. wird von Browsern oder separaten _Runtimes_ implementiert \\
	}

	\begin{itemize}
		\item<1-> Sicheres, portables, kompaktes und effizientes Format
		\item<1-> Hauptsächlich für leistungsstarke \cemph{Webanwendungen}
		\item<2-> Alleinstehende \cemph{Spezifikation}
		\item<3-> Implementation durch \cemph{Browser} oder separate \cemph{Runtimes}
	\end{itemize}
\end{frame}

\begin{frame}{Beispiel Ein-/Ausgabe}
	\pdfpcnote{
		@Silas \\
		\\
		- selbes Fibonacci Beispiel \\
		- Ausgabe ist eine Binaerdatei \\
		- Kompakt aber unuebersichtlich und unleserlich \\
	}

	\raggedleft
	\begin{minipage}{0.45\textwidth}
		\Lirsting[float=H, fancyvrb={frame=none, fontsize=\small}]{deps/paper/listings/fib.rush}
	\end{minipage}
	\begin{minipage}{0.35\textwidth}
		\raggedright
		\Darrow{Ausgabe}
	\end{minipage}

	\centering
	\begin{minipage}{0.7\textwidth}
		\Lirsting[float=H, fancyvrb={frame=none, fontsize=\footnotesize, numbers=none}]{listings/fib_wasm.hexdump}
	\end{minipage}
\end{frame}

\begin{frame}{Textdarstellung}
	\pdfpcnote{
		@Silas \\
		\\
		- Alternative Textdarstellung \\
		- aehnlich zu gewoehnlichem Assembly \\
	}
	\centering
	\begin{minipage}{0.67\textwidth}
		\Lirsting[float=H, ranges={1-1, 5-32}, fancyvrb={frame=none, fontsize=\scriptsize}]{listings/fib.wat}
	\end{minipage}
\end{frame}

\newcommand{\TableCell}[2]{\begin{minipage}{5cm}\Lirsting[float=H, fancyvrb={frame=none, numbers=none, fontsize=\footnotesize}, ranges={#2}]{listings/wasm_table.#1}\end{minipage}}

\begin{frame}{Hoher Abstraktionsgrad}
	\pdfpcnote{
		@Silas \\
		\\
		- Strukturen mit hohem Abstraktionsgrad werden bereitgestellt \\
		- Beispiele: (links rush, rechts WebAssembly) \\
		\\
		--- \\
		\\
		1. Funktionen \\
		2. if-else Verzweigungen \\
		3. Schleifen \\
	}

	\centering
	\rowcolors{2}{gray!15}{}
	\begin{tabular}{l|l}
		\rowcolor{gray!20} rush & WebAssembly            \\
		\hline
		\TableCell{rush}{1-3}   & \TableCell{wat}{1-5}   \\
		\TableCell{rush}{5-9}   & \TableCell{wat}{7-12}  \\
		\TableCell{rush}{11-13} & \TableCell{wat}{14-17} \\
	\end{tabular}
\end{frame}
