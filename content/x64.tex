\section{Kompilierung zu x86\_64}
\begin{frame}{Einleitung}
	\begin{itemize}
		\item \TODO{@RubixDev Write this}
	\end{itemize}
\end{frame}

\begin{frame}{Fazit zu \riscv}
	\begin{table}[h]
		\caption{Fazit zu x64}\label{tbl:x64_fazit}
		\rowcolors{2}{gray!15}{}
		\begin{tabularx}{0.95\textwidth}{LL}
			\cellcolor{green!20} Vorteile                       & \cellcolor{red!20} Nachteile                            \\ \hline
			Sehr neu und modern                                 & geringe Verbreitung                                     \\ \hline
			komplett open-source und Gemeinschaftlich verwaltet & eher experimentell                                      \\ \hline
			Sehr übersichtliche und simple Architektur          & einige Operationen sind aufwendiger                     \\ \hline
			Weniger Online-Ressourcen                           & sehr gute und übersichtliche Dokumentation \\ \hline
		\end{tabularx}
	\end{table}
\end{frame}

\begin{frame}{Fazit zu x64}
	\begin{table}[h]
		\caption{Fazit zu x64}\label{tbl:x64_fazit}
		\rowcolors{2}{gray!15}{}
		\begin{tabularx}{0.95\textwidth}{LL}
			\cellcolor{green!20} Vorteile       & \cellcolor{red!20} Nachteile                                    \\ \hline
			höherer Abstraktionsgrad als \riscv & Kompliziertere Übersetzung von z.B. Division und Multiplikation \\ \hline
			Weite Verbreitung                   & Sehr alt und unübersichtlich                                    \\ \hline
			Viele Online-Ressourcen             & Weniger übersichtliche Dokumentation                            \\ \hline
		\end{tabularx}
	\end{table}
\end{frame}
