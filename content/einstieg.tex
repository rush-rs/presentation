\section{TODO}
\begin{frame}{TODO}
	\begin{itemize}
		\item \TODO{translate everything into German}
		\item \TODO{fix lirstings captions}
	\end{itemize}
\end{frame}

\section{Einstieg \& Motivation}
\begin{frame}{Einstieg \& Motivation}
	\begin{itemize}
		\item Computerprogramme werden oft in speziell entwickelten Programmiersprachen verfasst
		\item Vorteile eines hohen Abstraktionsgrades~\scite[S.~9]{Dandamudi2005Risc}:
		      \begin{itemize}
			      \item Entwicklung ist einfacher und schneller
			      \item Programme sind portabel
			      \item Reperaturen und Erweiterungen sind einfacher
		      \end{itemize}
	\end{itemize}
\end{frame}

\begin{frame}{Zentrales Problem}
	\begin{minipage}{.35\textwidth}
		\Lirsting[float=H, fancyvrb={frame=none, fontsize=\small}]{deps/paper/listings/simple.rush}
	\end{minipage}%
	\Larrow{Übersetzung}
	\hfill
	\begin{minipage}{.35\textwidth}
		\Lirsting[float=H, fancyvrb={frame=none, fontsize=\small}, ranges={1-10}]{listings/rush_simple.hexdump}
	\end{minipage}

	\begin{itemize}
		\item Programme sollten \cemph{einfach} zu schreiben sein
		\item[\Rightarrow] Ein Computer muss diese jedoch auch \cemph{einfach} verarbeiten
	\end{itemize}
\end{frame}

\begin{frame}{Methoden zur Programmausführung}
	\begin{itemize}
		\item Man unterscheidet zwischen \cemph{Compilern} und \cemph{Interpretern}
		\item Ein Compiler übersetzt die Sprache in ein zielspezifiesches Format, welches der Computer versteht
		\item Ein Interpreter führt das Programm direkt aus, ohne es vorher zu übersetzen
	\end{itemize}
\end{frame}

\begin{frame}{Etappen der Übersetzung~\scite[S.~6--7]{wirth_compiler_construction_2005}}
	\hspace{0pt} % This is somehow required
	\vfill
	\begin{figure}[h]
		\begin{adjustbox}{max totalsize={\textwidth}{!},center}
			\begin{tikzpicture}[node distance=1cm, inner sep=3mm]
				\node (lexical_analysis) [rec, minimum height=1.5cm] {Lexikalische Analyse};
				\node (syntactic_analysis) [rec, right=of lexical_analysis, align=center, minimum height=1.5cm] {Syntaxanalyse};
				\draw [arrow] (lexical_analysis) -- (syntactic_analysis);
				\node (semantic_analysis) [rec, right=of syntactic_analysis, align=center, minimum height=1.5cm] {Semantische\\Analyse};
				\draw [arrow] (syntactic_analysis) -- (semantic_analysis);
				\node (codegen) [rec, right=of semantic_analysis, minimum height=1.5cm] {Code-Erzeugung};
				\draw [arrow] (semantic_analysis) -- (codegen);
			\end{tikzpicture}
		\end{adjustbox}
		\caption{Etappen der Übersetzung.}\label{fig:compilation_steps}
	\end{figure}
	\vfill
	\begin{figure}[h]
		\begin{adjustbox}{max totalsize={\textwidth}{!},center}
			\begin{tikzpicture}[node distance=3mm and 1cm, inner sep=3mm]
				\node (syntactic_analysis_text) [inner sep=0] {Syntaxanalyse};
				\node (lexical_analysis) [rec, below=of syntactic_analysis_text] {Lexikalische Analyse};
				\node (syntactic_analysis) [rec, fit={(syntactic_analysis_text) (lexical_analysis)}] {};
				\node (semantic_analysis) [rec, right=of syntactic_analysis] {Semantische Analyse};
				\draw [arrow] (syntactic_analysis) -- (semantic_analysis);
				\node (codegen) [rec, right=of semantic_analysis] {Code-Erzeugung};
				\draw [arrow] (semantic_analysis) -- (codegen);
			\end{tikzpicture}
		\end{adjustbox}
		\caption{Etappen der Übersetzung (angepasst).}\label{fig:compilation_steps_altered}
	\end{figure}
	\vfill
\end{frame}

\subsection{Die Programmiersprache \enquote{rush}}

\begin{frame}{Fähigkeiten von rush}
	\begin{table}[h]
		\caption{Die wichtigsten Fähigkeiten von rush.}\label{tbl:rush_features}
		\rowcolors{2}{gray!15}{}
		\begin{tabularx}{0.95\textwidth}{Ll}
			\rowcolor{gray!25} Bezeichnung & Beispiel                                          \\
			\hline
			Schleife                       & \LirstInline{rush}{loop {  }}                     \\
			while-Schleife                 & \LirstInline{rush}{while x < 5 {  }}              \\
			for-Schleife                   & \LirstInline{rush}{for i = 0; i < 5; i += 1 {  }} \\
			if-Verzweigung                 & \LirstInline{rush}{if true {  } else {  }}        \\
			Funktionsdefinition            & \LirstInline{rush}{fn foo(n: int) {  }}           \\
			infix-expression               & \LirstInline{rush}{1 + n; 5 ** 2}                 \\
			prefix-expression              & \LirstInline{rush}{!false; -n}                    \\
			let-statement                  & \LirstInline{rush}{let mut answer = 42}           \\
			cast-expression                & \LirstInline{rush}{42 as float}                   \\
		\end{tabularx}
	\end{table}
\end{frame}

\begin{frame}{Datentypen in rush}
	\begin{table}[h]
		\caption{Datentypen in rush.}\label{tbl:rush_types}
		\rowcolors{2}{gray!15}{}
		\begin{tabularx}{0.95\textwidth}{lL}
			\rowcolor{gray!25} Bezeichnung & Instanziierung einer Variable            \\
			\hline
			\qVerb{int}                    & \LirstInline{rush}{let a: int = 0;}      \\
			\qVerb{float}                  & \LirstInline{rush}{let b: float = 3.14;} \\
			\qVerb{bool}                   & \LirstInline{rush}{let c: bool = true;}  \\
			\qVerb{char}                   & \LirstInline{rush}{let d: char = 'a';}   \\
			\qVerb{()}                     & \LirstInline{rush}{let e: () = main();}  \\
			\qVerb{!}                      & \LirstInline{rush}{let f = exit(42);}    \\
		\end{tabularx}
	\end{table}
\end{frame}

\begin{frame}{Beispiel Programmtext in rush}
	\center
	\begin{minipage}{.95\textwidth}
		\Lirsting[caption={Beispiel: Berechnung von Fibonaccizahlen in rush.}, label={lst:rush_fib}, fancyvrb={frame=none}, float=H]{deps/paper/listings/fib.rush}
	\end{minipage}
\end{frame}

\begin{frame}{Fakten über rush}
	\begin{itemize}
		\item Im Git Commit `\rushCommit'~umfasste das Projekt \tokei{./deps/rush} Zeilen Programmtext\footnote{Leerzeilen und Kommentare werden nicht gezählt.}
		\item Das Projekt enthält zwei Interpreter, einen Transpiler und vier Compiler
		\item Die Komponenten verwenden alle die selbe Semantikanalyse und den selben Lexer und Parser
	\end{itemize}
\end{frame}

\begin{frame}{Programmtext der einzelnen Komponenten}
	\begin{table}[h]
		\centering
		\caption{Zeilen Programmtext pro Komponente.}\label{tbl:rush_loc_components}
		\begin{tabularx}{0.95\textwidth}{|L|L|}
			\hline
			\rowcolor{gray!20} Komponente & Zeilen Programmtext                                                  \\ \hline
			Lexer / Parser                & \tokei{./deps/rush/crates/rush-parser}                               \\ \hline
			Tree-walking interpreter      & \cellcolor{gray!30} \tokei{./deps/rush/crates/rush-interpreter-tree} \\ \hline
			VM compiler / runtime         & \tokei{./deps/rush/crates/rush-interpreter-vm}                       \\ \hline
			WASM compiler                 & \tokei{./deps/rush/crates/rush-compiler-wasm}                        \\ \hline
			LLVM compiler                 & \cellcolor{gray!30} \tokei{./deps/rush/crates/rush-compiler-llvm}    \\ \hline
			RISC-V compiler               & \tokei{./deps/rush/crates/rush-compiler-risc-v}                      \\ \hline
			x86 compiler                  & \cellcolor{gray!30} \tokei{./deps/rush/crates/rush-compiler-x86-64}  \\ \hline
		\end{tabularx}
	\end{table}
\end{frame}
