% \section{Einstieg \& Motivation}
% \begin{frame}{Einstieg \& Motivation}
% 	\begin{itemize}
% 		\item Computerprogrammer werden oft in speziell entwickelten Programmiersprachen verfasst
% 		\item Vorteile eines hohen Abstraktionsgrades~\cite[S.~9]{Dandamudi2005Risc}:
% 		      \begin{itemize}
% 			      \item Entwicklung ist einfacher und schneller
% 			      \item Programme sind portabel
% 			      \item Instandhaltung ist einfacher
% 		      \end{itemize}
% 		\item Programme sollten einfach zu schreiben sein
% 		\item[\Rightarrow] Zentrales Problem: Ein Computer muss diese jedoch auch verstehen
% 	\end{itemize}
% \end{frame}
%
% \begin{frame}{Methoden zur Programmausführung}
% 	\begin{itemize}
% 		\item Man unterscheidet zwischen \emph{Compilern} und \emph{Interpretern}
% 		\item Ein Compiler (auch \emph{Übersetzer}) übersetzt die Sprache in ein zielspezifiesches Format, sodass ein Computer dieses verstehen kann
% 		\item Ein Interpreter führt das Programm direkt aus, ohne es vorher zu bearbeiten
% 	\end{itemize}
% \end{frame}

\begin{frame}{Etappen der Übersetzung}
	\begin{figure}[h]
		\begin{adjustbox}{max totalsize={\textwidth}{!},center}
			\begin{tikzpicture}[node distance=1cm, inner sep=3mm]
				\node (lexical_analysis) [rec] {lexical analysis};
				\node (syntactic_analysis) [rec, right=of lexical_analysis, align=center] {syntactical\\analysis};
				\draw [arrow] (lexical_analysis) -- (syntactic_analysis);
				\node (semantic_analysis) [rec, right=of syntactic_analysis, align=center] {semantic\\analysis};
				\draw [arrow] (syntactic_analysis) -- (semantic_analysis);
				\node (codegen) [rec, right=of semantic_analysis] {code generation};
				\draw [arrow] (semantic_analysis) -- (codegen);
			\end{tikzpicture}
		\end{adjustbox}
		\caption{Etappen der Übersetzung}{\cite[S.~6--7]{wirth_compiler_construction_2005}}\label{fig:compilation_steps}
	\end{figure}
\end{frame}

\begin{frame}{Etappen der Übersetzung (angepasst)}
	\begin{figure}[h]
		\begin{adjustbox}{max totalsize={\textwidth}{!},center}
			\begin{tikzpicture}[node distance=3mm and 1cm, inner sep=3mm]
				\node (syntactic_analysis_text) [inner sep=0] {syntactical analysis};
				\node (lexical_analysis) [rec, below=of syntactic_analysis_text] {lexical analysis};
				\node (syntactic_analysis) [rec, fit={(syntactic_analysis_text) (lexical_analysis)}] {};
				\node (semantic_analysis) [rec, right=of syntactic_analysis] {semantic analysis};
				\draw [arrow] (syntactic_analysis) -- (semantic_analysis);
				\node (codegen) [rec, right=of semantic_analysis] {code generation};
				\draw [arrow] (semantic_analysis) -- (codegen);
			\end{tikzpicture}
		\end{adjustbox}
		\caption{Etappen der Übersetzung (angepasst)}{\cite[S.~6--7]{wirth_compiler_construction_2005}}\label{fig:compilation_steps_altered}
	\end{figure}
\end{frame}

\subsection{Die Programmiersprache \enquote{rush}}

\begin{frame}{Die Programmiersprache \enquote{rush}}
	\Lirsting[caption={Erzeugen von Fibonaccizahlen in rush}, label={lst:rush_fib}, float=H]{deps/paper/listings/fib.rush}
\end{frame}

\begin{frame}{Fähigkeiten von rush}
	\begin{table}[h]
		\caption{Die wichtigsten Fähigkeiten von rush}\label{tbl:rush_features}
		\rowcolors{2}{gray!15}{}
		\begin{tabularx}{0.95\textwidth}{lL}
			\rowcolor{gray!25} Name & rush Code                                         \\
			\hline
			Loop                    & \LirstInline{rush}{loop {  }}                     \\
			While                   & \LirstInline{rush}{while x < 5 {  }}              \\
			For                     & \LirstInline{rush}{for i = 0; i < 5; i += 1 {  }} \\
			If                      & \LirstInline{rush}{if true { /* */ } else {  }}   \\
			Fn                      & \LirstInline{rush}{fn foo(n: int) {  }}           \\
			Infix Expr              & \LirstInline{rush}{1 + n; 5 ** 2}                 \\
			Prefix Expr             & \LirstInline{rush}{!false; -n}                    \\
			Variables               & \LirstInline{rush}{let mut answer = 42}           \\
			Cast Expr               & \LirstInline{rush}{42 as char}                    \\
		\end{tabularx}
	\end{table}
\end{frame}

\begin{frame}{Datentypen in rush}
	\begin{table}[h]
		\caption{Datentypen in rush}\label{tbl:rush_types}
		\rowcolors{2}{gray!15}{}
		\begin{tabularx}{0.95\textwidth}{lL}
			\rowcolor{gray!25} Notation & Example Data                             \\
			\hline
			\qVerb{int}                 & \LirstInline{rush}{let a: int = 0;}      \\
			\qVerb{float}               & \LirstInline{rush}{let b: float = 3.14;} \\
			\qVerb{bool}                & \LirstInline{rush}{let c: bool = true;}  \\
			\qVerb{char}                & \LirstInline{rush}{let c: char = 'a';}   \\
			\qVerb{()}                  & \LirstInline{rush}{let d: () = main();}  \\
			\qVerb{!}                   & \LirstInline{rush}{let e = exit(42);}    \\
		\end{tabularx}
	\end{table}
\end{frame}

\begin{frame}{Fakten über rush}
    \begin{itemize}
        \item Im Commit \rushCommit~umfasste das Projekt \rushLoc~Zeilen Programmtext\footnote{Leerzeilen und Kommentare werden nicht gezählt}
        \item Enthält einen Lexer, einen Parser, fünf Compiler und einen Interpreter
    \end{itemize}
\end{frame}
